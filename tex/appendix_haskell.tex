\section{An algorithm for unranking m\'enage permutations}
\label{apndx:haskell}
In this section, we will provide Haskell implementations of the functions in
Chapter \ref{cha:UnrankingMenage} that were used to compute the permutations
requested by Richard Arratia's bounties.
Each code sample will reference its
precedent in Chapter \ref{cha:UnrankingMenage}. We start by importing some
relevant libraries
and establishing some naming conventions.

\begin{singlespace}\begin{verbatim}
import Data.List (sort, nub, intersect)

type Prefix = [Int]
type Coefficients = [Integer]
type PrefixCount = Prefix -> Integer
\end{verbatim}\end{singlespace}

\subsection{Unranking from prefix counts}

We begin with the function \texttt{unrank}, which implements
$\operatorname{unrank}_\mathcal{C}(i)$, as shown in
Theorem \ref{theorem:unrankFromPrefix}. The internal function
\texttt{recurse} corresponds to the recursive function $f_i^\mathcal{C}$.
\begin{singlespace}\begin{verbatim}
unrank :: Int ->                 -- Alphabet of n letters
          Int ->                 -- Words of length k
          (Prefix -> Integer) -> -- #prefix function
          Integer ->             -- Unrank at i
          Prefix                 -- Word at rank i
unrank n k prefixCounter i = recurse (0, 0) 1 [] where
  recurse :: (Integer, Integer) -> -- index range for prefix (a, b]
              Int ->                -- candidate for current letter
              Prefix ->             -- established prefix
              Prefix                -- word at index
  recurse (a, b) c prefix
    | c > n              = error "Out of range!"
    | length prefix == k = prefix
    | a < i && i <= b    = recurse (a, b') 1 (prefix  ++ [c])
    | otherwise          = recurse (b, b'') (c + 1) prefix where
    b'      = a + prefixCounter (prefix ++ [c, 1])
    b''     = b + prefixCounter (prefix  ++ [c + 1])
\end{verbatim}\end{singlespace}

\subsection{Counting derangements with prefixes}
This part of the code contains three functions. Each of them takes a parameter
$n$, which indicates the number of letters in the derangement.
The first function, \texttt{invalidDerangementPrefix}, also takes a prefix
$\alpha$ and checks to see if it includes duplicate values or has
letters in forbidden positions.
The function \texttt{derangementPrefixCount}
is equivalent to the function $\#\operatorname{prefix}_\mathcal{D}$,
as described in Corollary \ref{cor:derangementsWithPrefix}.
The function \texttt{unrankDerangement} is equivalent to the function
$\operatorname{unrank}_\mathcal{D}$ as described in
Theorem \ref{theorem:unrankFromPrefix}.
\begin{singlespace}\begin{verbatim}
invalidDerangementPrefix :: Int -> Prefix -> Bool
invalidDerangementPrefix _ prefix = containsDupes || invalidPosition where
  containsDupes = prefix /= (nub prefix)
  invalidPosition = any (id) $ zipWith (==) [1..] prefix

derangementPrefixCount :: Int -> Prefix -> Integer
derangementPrefixCount n prefix
  | invalidDerangementPrefix n prefix = 0
  | otherwise = sum $ map (\j -> term j) [0..squareCount] where
    k = length prefix
    squareCount = fromIntegral $ n - k - length (intersect [k+1..n] prefix)
    n' = fromIntegral n
    k' = fromIntegral k
    term j = (-1)^j * binomial squareCount j * factorial (n' - k' - j)

unrankDerangement :: Int -> Integer -> Prefix
unrankDerangement n = unrank n n (derangementPrefixCount n)
\end{verbatim}\end{singlespace}

\subsection{Decomposition of derived complementary boards}
These next two functions decompose a derived complementary board into
staircase-shaped blocks.
The first function, \texttt{getColGroups}, corresponds to the partition
into contiguous parts referred to in Lemma \ref{lem:subBoardSizes}.
The second function, \texttt{blockSizesFromPrefix} goes on to implement the
sum in this lemma.
\begin{singlespace}\begin{verbatim}
getColGroups :: Int -> Prefix -> [[Int]]
getColGroups n prefix = filter (not . null) $ colGroups where
  cols = 0 : (sort prefix) ++ [n+1]
  colGroups = zipWith (\a b ->  [a+1..b-1]) cols (tail cols)

blockSizesFromPrefix :: Int -> Prefix -> [Int]
blockSizesFromPrefix n prefix = map (sum . map colCells) colGroups where
  colGroups = getColGroups n prefix
  k = length prefix
  colCells c
    | c < k      = 0
    | c == k     = 1
    | c == n     = 1
    | otherwise  = 2
\end{verbatim}\end{singlespace}

\subsection{Complementary polynomials}
Since the functions above compute the size of the staircase-shaped
sub-boards, they can hand these values to \texttt{fibonacciPoly}, which
implements Fibonacci polynomials as described in
Definition \ref{def:FibonacciPolynomial}. These match the rook polynomials of
the disjoint staircase-shaped sub-boards, so in
\texttt{complementaryRookPolynomial}, the appropriate polynomials are multiplied
together and the coefficients are extracted to implement
Lemma \ref{lemma:CountsFromComplementaryPolynomials}.
\begin{singlespace}\begin{verbatim}
fibonacciPoly :: Int -> Coefficients
fibonacciPoly = (!!) fibonacciPolynomials where
  fibonacciPolynomials = [1] : [1] : recurse [1] [1] where
    recurse f g = h : recurse g h where
      h = ([0,1] .*. f) .+. g

complementaryRookPolynomial :: Int -> Prefix -> Coefficients
complementaryRookPolynomial n prefix = foldr (.*.) [1] blockPolys where
  blockPolys = map (\i -> fibonacciPoly (i + 1)) blockSizes where
    blockSizes = blockSizesFromPrefix n prefix
\end{verbatim}\end{singlespace}

\subsection{Putting it together}
This part of the code contains three functions.
Each takes a parameter $n$, which indicates the number of letters in the
m\'enage permutation.
The first function, \texttt{invalidMenagePrefix}, also takes a prefix
$\alpha$ and checks to see if it includes duplicate values or has
letters in forbidden positions.
The function \texttt{menagePrefixCount}
is equivalent to the function $\#\operatorname{prefix}_\pazocal{M}$,
as described in Corollary \ref{cor:derangementsWithPrefix}.
The function \texttt{unrankDerangement} is equivalent to the function
$\operatorname{unrank}_\pazocal{M}$ as described in
Theorem \ref{theorem:unrankFromPrefix}.
\begin{singlespace}\begin{verbatim}
invalidMenagePrefix :: Int -> Prefix -> Bool
invalidMenagePrefix n prefix = containsDuplicates || invalidPosition where
  containsDuplicates = prefix /= (nub prefix)
  invalidPosition = any inRestrictedPosition $ zip [0..] prefix where
    inRestrictedPosition (i, x) = (x `mod` n == i) || (x == i + 1)

menagePrefixCount :: Int -> Prefix -> Integer
menagePrefixCount n prefix
  | invalidMenagePrefix n prefix = 0
  | otherwise      = recurse 0 crp where
  n' = fromIntegral (n - length prefix)
  crp = complementaryRookPolynomial n prefix
  recurse k (c:cs) = (-1)^k * c * factorial (n'-k) + recurse (k+1) cs
  recurse _ [] = 0

unrankMenage :: Int -> Integer -> Prefix
unrankMenage n = unrank n n (menagePrefixCount n)
\end{verbatim}\end{singlespace}

\subsection{Miscellanea}
These are helper functions that need to be included in order for the
code to compile. The first two functions add and multiply list representations
of polynomials, and the third and fourth function implement the factorial
function and the binomial coefficient.
\begin{singlespace}\begin{verbatim}
-- The polynomial a + bx + cx^2 ... is represented as
-- [a, b, c, ...]
-- These are helper functions for adding and multiplying polynomials
(.+.) :: Coefficients -> Coefficients -> Coefficients
(.+.) p1 [] = p1
(.+.) [] p2 = p2
(.+.) (a:p1) (b:p2) = (a + b) : (p1 .+. p2)

(.*.) :: Coefficients -> Coefficients -> Coefficients
(.*.) p1 [] = []
(.*.) [] p2 = []
(.*.) p1 p2 = foldr1 (.+.) termwiseProduct where
  termwiseProduct = map f $ zip [0..] p1 where
    f (i, x) = replicate i 0 ++ map (*x) p2

factorial :: Integer -> Integer
factorial n = factorial' n 1 where
  factorial' 0 accum = accum
  factorial' n accum = factorial' (n - 1) (accum * n)

binomial :: Integer -> Integer -> Integer
binomial _ 0 = 1
binomial 0 _ = 0
binomial n k
  | n < k'    = 0
  | otherwise = product [k' + 1..n] `div` factorial (n - k') where
    k' = max k (n - k)
\end{verbatim}\end{singlespace}
