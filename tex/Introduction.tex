\chapter{Introduction}
\label{cha:introduction}

For me, the best problem is one that could be understood by a high school
student, but whose solution cleverness, and techniques that one could only
know by standing on the shoulders of centuries of mathematicians. Broadly
speaking, my mathematical interests might be classified as recreational
mathematics, and I hope that---at least at a high level---this dissertation
is as inviting as that classification suggests.

\section{Elevator pitches}

This first section of the Introduction is aimed at doing just that:
describing the three main chapters of this dissertation in informal terms.
My target audience for this section is a friend, a family member,
a student, or a curious stranger.

In this dissertation---as the title promises---we will look at
permutations, sta\-tis\-tics, and switch\-es, although not in that order! Here we
give a high-level overview of what all of this is about.

\subsection{Spinning switches, informally}

Chapter \ref{cha:SpinningSwitches} is about generalizing an old puzzle
popularized by Marin Gardner in the 1970's. My favorite version of this
puzzle comes from Peter Winkler and is described at the beginning of the
chapter. (We will repeat it here for convenience.)

\begin{quote}
  Four identical, unlabeled switches are wired in series to a light bulb.
  The switches are simple buttons whose state cannot be directly observed,
  but can be changed by pushing; they are mounted on the corners of a
  rotatable square. At any point, you may push, simultaneously, any subset
  of the buttons, but then an adversary spins the square. Show that there
  is a deterministic algorithm that will enable you to turn on the bulb in
  at most some fixed number of steps. \cite{Winkler2004}
\end{quote}

(This is a fun puzzle to work out on your own, but you can also skip to
the proof of Proposition \ref{prop:WinklersSolution} to see the solution.)

We can generalize this puzzle in at least two ways.
The first way we can generalize the puzzle is by asking about different kinds of
switches, for instance one could imagine a dial with three states,
and only one of them is on. On any turn we could rotate it one or two
clicks to put it in one of the other states.
(If we rotate it zero or three clicks, it will be in the same state as before.)
We consider examples of two other types of switches,
with $6$ states and with $8$ states, in Example \ref{ex:S3D8Schematics}.
In general, we will want to solve the problem for switches that can behave like
arbitrary \textit{finite groups}.

The second way we can generalize the puzzle is by looking at the number of
switches and how they ``spin''. For example, we could have six switches on
the corners of a rotating hexagon. As another example, we could have three
switches that can be scrambled however the adversary pleases.
We can model the adversary's scrambling as a \textit{faithful group action}
on a finite set.

We want to know what combinations of ``switches'' and ``spinning''
result in a puzzle that our puzzle-solver can solve within a finite number of
steps.

\subsection{Permutations with a given number of \texorpdfstring{$k$}{k}-cycles, informally}
Chapter \ref{cha:PermutationStatistics} concerns permutations,
which we can informally think of ways of
scrambling the whole numbers from $1$ up to $n$. We're interested in certain
kinds of permutations, for example, those where $i$ is never in the $i$-th
position. Another kind of permutation we might be interested in is those with
a certain number of \textit{transpositions}, which means the number of unordered
pairs $\{i, j\}$ such that $i$ is in position $j$ and $j$ is in position $i$.
(i.e. $i$ and $j$ ``swap places'').

Now if we look at all of the permutations on $n$ letters with $k$ transpositions,
and then we take a look at the average of all of the first letters, we see
something unusual: this number is closely related to ways of rotating or
reflecting a square, cube, or higher dimensional \textit{hypercube}
in such a way that all of the sides of the square, faces of the cube, or
\textit{facets} of the hypercube move to a new position.

A priori, we should be surprised that permutations with a given number of
transpositions have anything to do with moving the faces of high dimensional
cubes. This chapter seeks to explore why these objects are connected, and
to use this correspondence to understand both permutations and hypercubes a
little bit better.

\subsection{Unranking restricted permutations, informally}
Chapter \ref{cha:UnrankingMenage} also explores permutations. Here we look at
two kinds of permutations: \textit{derangements}, where the letter $i$ is never
in position $i$, and \textit{m\'enage} permutations, where $i$ is never in
position $i$ or in position $i - 1$.
(This ``wraps around'', so that $1$ cannot be in the last position either.)

Without too much trouble, we could start writing these down in
\textit{lexicographic} order (essentially, alphabetical order). The
problem is that the number of these permutations grows very quickly! For
$n=20$ there are more than $895$ quadrillion derangements and
$312$ quadrillion m\'enage permutations.

In 2020 and 2021, Richard Arratia posed two problems, each with \$100 bounties
attached. The first asking for the $500$ quadrillionth derangement. The
second for the $100$ quadrillionth m\'enage permutation.
With numbers this large, it's not feasible for a computer to
write them all down and simply pick the permutation in the desired position.
Instead I developed an algorithm that would allow me to compute these by hand in
a few hours, with a handheld calculator in a few minutes, or with a computer
program in a few milliseconds.

This chapter explains these techniques that I developed. These techniques are
described using \textit{rook theory}, which is a mathematical way of analyzing
the number of ways rooks can be placed on a chessboard, subject to certain
restrictions.

\section{Motivations and overview}
% ----------------------------------------------------------------------------
This section is aimed at mathematicians who know some of the jargon, and want
a more formal (but still high-level) preview of what comes next.

\subsection{Overview of spinning switches}
Chapter \ref{cha:SpinningSwitches}, we discuss a broad generalization of a
puzzle popularized by Martin Gardner. The puzzles that I call
``generalized spinning switches puzzles'' are variations on the following theme:
a puzzle-solver has four indistinguishable switches (or coins, or pint glasses)
on the corners of a rotatable square table. Without being able to see the state
of the switches, the puzzle-solver attempts to get them all in a given state.
However, an adversary spins the table between each move, which means that the
puzzle-solver is playing with imperfect information.

I first learned of this puzzle from William professor Steve Miller's
``Math Riddles'' website.
He had a variation of the puzzle listed under the name ``Don't Flip Out, Square!''
I thought about the puzzle, solved it, brought this puzzle to the 2019 Graduate
Student Combinatorics Colloquium to share it with other participants. There,
we attempted to solve generalizations of the puzzle.

In generalized spinning switches, we have switches that behave like arbitrary
groups $G$, and spinning that allows the adversary to permute the switches in
ways prescribed by an arbitrary group $H$. We use the abstraction of a wreath
product to neatly package these two groups into a puzzle.
Other authors have considered certain families of such switches, such as cyclic
groups, but in this chapter we resolve the puzzle for all groups of prime order
other kinds of groups, including for switches that behave like the monster
group, the largest of the sporadic finite simple groups.

Along the way, we provide reductions for proving that certain kinds of puzzles
do not have solutions, and decompositions for constructing winning strategies
from simpler puzzles.

\subsection{Overview of permutations with a given number of \texorpdfstring{$k$}{k}-cycles}
The topic in Chapter \ref{cha:PermutationStatistics}
came out of a discussion that my advisor and I had about a delightful
theorem of Mark Conger, where he proved that the expected value of $\pi(1)$
over all permutations $\pi$ with $k$ descents is $k + 1$---and that this doesn't
depend on the number of letters in the permutation.

Curious, I generated pages of tables of numerical data about the expected value
of the first letter of permutations $\pi \in S_n$ such that
$\operatorname{stat}(\pi) = k$ for various permutation statistics.
Most didn't appear to have much structure making conjectures hard to form,
and others had \textit{too much} structure, making conjectures too easy to
prove.
But the permutation statistic that counted the number of transpositions in a
permutation was in the Goldilocks zone: the denominators of these average
values appeared to be consistent as I scanned diagonally across the table.
When I looked these up in the On-Line Encyclopedia of Integer sequences,
they were found to correspond to
``$(n-1)$-dimensional facet derangements for the $n$-dimensional hypercube.''

When I looked at the permutation statistic that counted $k$-cycles instead of
transpositions, the tables presented a similar structure. This time, the
denominators corresponded to the number of derangements of the generalized
symmetric group. Ultimately I developed a full characterization of the four
parameter family: the expected value of the $\ell$-th letter over all
permutations $\pi \in S_n$ with exactly $m$ $k$-cycles.
A formula for this expected value is given both inn terms of a sum,
and---more satisfyingly---in terms of the number of derangements of the
 generalized symmetric group.

Along the way, I was distracted by a pairs of sets of equal cardinality,
and I found that I couldn't resist searching for a bijection.
In particular, the cardinality of sets implied that there was a
family of reversible insertion algorithms that preserved the number of
$k$-cycles in most cases. A simple bijection proved to be elusive, but
the recursive insertion algorithm (and its inverse) are developed in
Section \ref{section:bijection} of that chapter.

\subsection{Overview of unranking restricted permutations}
In chapter \ref{cha:UnrankingMenage}, we develop the solution of two related
problems posed by Richard Arratia, each with a bounty attached.
These problems are particular examples of a broader class of so called
\textit{unranking} problems:
given a total order on a set of combinatorial objects (that can be counted
efficiently), when is it possible to efficiently compute the $k$-th smallest
element?

In this chapter, we look at two families of restricted permutations:
derangements and m\'enage permutations on $n$ letters, both ordered
lexicographically as words.
We start by developing a more general theory that allows us to unrank a set
of words whenever we can count the number of words with a given prefix.
Then we use ideas from rook theory to develop methods for counting the number
of permutations in each of these two families that have a given prefix.
Finally, we put these techniques together and provide an explicit, efficient
algorithms for unranking derangements and m\'enage permutations.

% Imagine a square table that rotates about its center. At each corner is a deep
% well, and at the bottom of each well is a drinking glass that is either upright
% or inverted. You cannot see into the wells, but you can reach into them and feel
% whether a glass is turned up or down. A move is defined as follows: Spin the
% table, and when it stops, put each hand into a different well. You may adjust
% the orientation of the glasses any way you like, that is, you may leave them as
% they are or turn one glass or both. Now, spin the table again and repeat the
% same procedure for your second move. When the table stops spinning, there is no
% ay to distinguish its corners, and so you have only two choices: you may reach
% into any diagonal pair of wells or into any adjacent pair. The object is to get
% all four glasses turned in the same way, either all up or all down. When this
% task is accomplished, a bell rings. At the start, the glasses in the four wells
% are turned up or down at random. If they all happen to be turned in the same
% direction at this point, the bell will ring at once and the task will have been
% accomplished before any moves were made. Therefore it should be assumed that at
% the start the glasses are not all turned the same way. It is also assumed that
% you are not allowed to keep your hands in the wells and make experimental
% turnings to see if the bell rings. Furthermore, you must announce in advance the
% two wells to be probed at each step. You cannot probe one well, then decide
% which other well to probe. Is there a procedure guaranteed to make the bell ring
% in a finite number of moves? Many people, after thinking briefly about this
% problem, conclude that there is no such procedure. It is a question of
% probability, they reason. With bad luck one might continue to make moves
% indefinitely. That is not the case, however. After no more than n correct moves
% one can be certain of ringing the bell. What is the minimum value of n, and what
% procedure is sure to make the bell ring in n or fewer moves?

% My dissertation consists of three discrete topics that are could each stand on their own.
