\section{Unranking combinatorial objects}
\label{apndx:unranking}
\subsection{Compositions}
\begin{definition}
  A \textbf{composition} of $n \in \mathbb{N}_{>0}$ is a sequence of
  strictly positive integers summing to $n$. We denote the set of compositions
  of n by $\pazocal C_n$.
\end{definition}

\begin{lemma}[\cite{Stanley2011EC1}] % (p. 18)
  The number of compositions of $n$ is $\#\pazocal C_n = 2^{n-1}$.
\end{lemma}

\begin{proposition}
  If $\alpha$ is a sequence of positive integers such that
  $\operatorname{sum}(\alpha) \leq n$, then
  the number of compositions of $n$ that start with the prefix $\alpha$
  is \begin{equation}
    \#\operatorname{prefix}_{\pazocal{C}_n}(\alpha) = \begin{cases}
      1 & \alpha \in \pazocal{C}_n \\
      2^{n - \operatorname{sum}(\alpha) - 1} & \text{otherwise}.
    \end{cases}
  \end{equation}
\end{proposition}

\begin{definition}
  For $k > 0$, a $k$\textbf{-composition} of $n \in \mathbb{N}_{>0}$
  is a sequence of length $k$ consisting of strictly positive integers that sum
  to $n$. We denote the set of $k$-compositions of $n$ by $\pazocal C_n^{k}$.
\end{definition}

\begin{lemma}[\cite{Stanley2011EC1}]
  The number of $k$-compositions of $n$ is $\#\pazocal C_n^{k} = \binom{n-1}{k-1}$.
\end{lemma}

\begin{proposition}
  If $\alpha \in \pazocal{C}_n^{k}$ or if $\alpha$ is
  a sequence of positive integers of length $\ell < k$ such that
  ${\operatorname{sum}(\alpha) < n}$,
  the number of $k$-compositions of $n$ that start with the prefix $\alpha$
  is \begin{equation}
    \#\operatorname{prefix}_{\pazocal{C}_n}(\alpha) = \begin{cases}
      1 & \alpha \in \pazocal{C}_n^{k} \\
      \binom{n - \operatorname{sum}(\alpha) - 1}{k - \ell - 1} & \text{otherwise}.
    \end{cases}
  \end{equation}
\end{proposition}

\subsection{Partitions}
\begin{definition}
  A \textbf{partition} of $n \in \mathbb{N}_{>0}$ is a weakly decreasing
  sequence of strictly positive integers summing to $n$. We denote the set of
  partitions of $n$ by $\pazocal P_n$.

  The set of partitions of $n$ with largest part at most
  $k$ is denoted $\pazocal P_n^{\leq k}$.
\end{definition}

\begin{lemma}[\cite{Stanley2011EC1}]
  The ordinary generating function for the number of partitions of $n$ is \begin{equation}
    \sum_{n=0}^\infty \#\pazocal P_n q^n = \prod_{j=1}^\infty \frac{1}{1-q^j},
  \end{equation}
  and the ordinary generating function for the number of partitions of $n$ with largest part
  as most $k$ is \begin{equation}
    \sum_{n=0}^\infty \#\pazocal P_n^{\leq j} q^n = \prod_{j=1}^k \frac{1}{1-q^j}.
  \end{equation}
\end{lemma}

\begin{proposition}
  If either $\alpha \in \pazocal{P}_n$ or $\alpha$ is
  a decreasing sequence of positive integers such that
  ${\operatorname{sum}(\alpha) < n}$ or
  then the number of partitions of $n$ that start with the prefix $\alpha$
  is \begin{equation}
    \#\operatorname{prefix}_{\pazocal{P}_n}(\alpha) = \begin{cases}
      1 & \alpha \in \pazocal{C}_n^{k} \\
      \#\pazocal P_{n-\operatorname{sum}(\alpha)}^{\leq \operatorname{min}(\alpha)} & \text{otherwise}.
    \end{cases}
  \end{equation}
\end{proposition}
